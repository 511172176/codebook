\documentclass{article}
\usepackage{listings}  % 用來格式化顯示程式碼
\usepackage{xcolor}    % 用來自定義程式碼顏色
\usepackage{xeCJK}     % 支援中文
\usepackage{fancyhdr}  % 自定義頁眉頁腳
\usepackage{multicol}  % 用來分欄顯示
\usepackage{geometry}  % 調整頁邊距
\usepackage{titlesec}  % 控制標題格式
\usepackage[utf8]{inputenc} % 使用 UTF-8 編碼
\usepackage{listingsutf8} % 支援UTF-8註解顯示

% 設置中文主字體
\setCJKmainfont{SimSun}

% 調整行距
\linespread{1.0}  % 保持正常行距

% 設置程式碼顯示樣式
\lstset{
  inputencoding=utf8, % 確保支持UTF-8編碼
  basicstyle=\ttfamily\small,               % 程式碼字體為 footnotesize
  keywordstyle=\bfseries\color{blue},              % 關鍵字樣式
  commentstyle=\itshape\scriptsize\color{green!50!black},  % 註解字體稍微縮小
  stringstyle=\color{red},                         % 字串顏色
  showstringspaces=false,                          % 不顯示空格
  numbers=left,                                    % 行號顯示在左邊
  numberstyle=\scriptsize\color{gray},             % 行號樣式
  stepnumber=1,                                    % 每行顯示行號
  numbersep=5pt,                                   % 行號與程式碼的間距
  breaklines=true,                                 % 開啟自動換行
  breakatwhitespace=true,                          % 在空白處換行
  tabsize=4,                                       % 設定 Tab 大小
  language=C++,                                    % 預設語言設為 C++
  columns=fullflexible                            % 減少字體間距,避免擠壓
}

% 設置頁邊距
\geometry{
  top=1in,
  bottom=1in,
  left=0.9in,    % 左邊邊距調整為 0.9in
  right=0.9in    % 右邊邊距調整為 0.9in
}

% 自定義頁眉和頁腳
\pagestyle{fancy}
\fancyhf{} % 清除預設頁眉頁腳
\fancyhead[L]{\fontsize{16}{18}\selectfont FJCU}  % 左側顯示學校名稱
\fancyhead[C]{\fontsize{16}{18}\selectfont Handmade}  % 調整頁眉中間的字體大小
\fancyhead[R]{\fontsize{16}{18}\thepage}  % 右上角顯示頁碼

% 調整標題格式,減少行與行之間的間距
\titleformat{\section}{\Large\bfseries}{}{0pt}{}
\titlespacing*{\section}{0pt}{10pt}{10pt}  % 控制 section 與上下內容的距離

% 設置分欄時的欄距和分隔線
\setlength{\columnsep}{35pt}  % 增大兩欄之間的欄距,設置為 35pt
\setlength{\columnseprule}{0.4pt}  % 設置兩欄之間的分隔線

\titleformat{\section}{\small\bfseries}{}{0pt}{} % 將 section 標題字體設為 \small
\titlespacing*{\section}{0pt}{10pt}{10pt}  % 可根據需要調整標題上下間距

\begin{document}

% 添加目錄
\footnotesize \tableofcontents  % 自動生成的目錄將在這裡顯示
\newpage  % 插入分頁,使目錄和正文分開
% 添加分欄,下面內容將分為兩欄顯示
\begin{multicols}{2}

\section{Tree Vector建樹, DFS先序尋訪, 找尋最近LCA共同祖先}

\lstinputlisting{C:/Users/ae887/OneDrive/桌面/codebook/code/vector建樹 先序尋訪 找尋最近共同祖先.cpp}

\section{Tree Disjoinset並查集, 路徑壓縮}

\lstinputlisting{C:/Users/ae887/OneDrive/桌面/codebook/code/disjoinset並查集路徑壓縮.cpp}

\section{Tree Fenwick tree鄰接表樹, 時間戳樹, 權值陣列, lowbit修改查詢區間和}

\lstinputlisting{C:/Users/ae887/OneDrive/桌面/codebook/code/disjoinset並查集路徑壓縮.cpp}

\section{Trie建樹, 修改, 查詢}

\lstinputlisting{C:/Users/ae887/OneDrive/桌面/codebook/code/Trie tree建樹, 修改, 查詢.cpp}

\section{Trie AC自動機, 多模式字串數}

\lstinputlisting{C:/Users/ae887/OneDrive/桌面/codebook/code/ac自動機, 多模式字串數.cpp}

\section{BST有序樹轉二元樹, 數組模擬樹}

\lstinputlisting{C:/Users/ae887/OneDrive/桌面/codebook/code/tree數組模擬樹.cpp}

\section{BST Stack後序轉二元樹}

\lstinputlisting{C:/Users/ae887/OneDrive/桌面/codebook/code/stack後序轉二元樹.cpp}

\section{BST前序加中序找出後序}

\lstinputlisting{C:/Users/ae887/OneDrive/桌面/codebook/code/tree前序中序找後序.cpp}

\section{BST後序二分搜尋樹還原}

\lstinputlisting{C:/Users/ae887/OneDrive/桌面/codebook/code/BST後序二分搜尋樹還原.cpp}

\section{BST建立結構指標二元樹, 前序尋訪}

\lstinputlisting{C:/Users/ae887/OneDrive/桌面/codebook/code/BST建立結構指標二元樹.cpp}

\section{BST Heap priority queue插入取出調整}

\lstinputlisting{C:/Users/ae887/OneDrive/桌面/codebook/code/Heap priority queue插入取出調整.cpp}

\section{BST Treap樹堆積左旋右旋, 插入刪除}

\lstinputlisting{C:/Users/ae887/OneDrive/桌面/codebook/code/Treap左旋右旋, 插入刪除.cpp}

\section{BST Treap rope結構操作字串修改Treap}

\lstinputlisting{C:/Users/ae887/OneDrive/桌面/codebook/code/Treap rope結構操作字串修改Treap.cpp}

\section{BST 霍夫曼樹最小代價Min Heap}

\lstinputlisting{C:/Users/ae887/OneDrive/桌面/codebook/code/BST 霍夫曼樹最小代價Min Heap.cpp}

\section{Graph BFS狀態空間搜尋最短路徑}

\lstinputlisting{C:/Users/ae887/OneDrive/桌面/codebook/code/Graph BFS狀態空間搜尋最短路徑.cpp}

\section{Graph DFS走訪, 建無向相鄰矩陣}

\lstinputlisting{C:/Users/ae887/OneDrive/桌面/codebook/code/Graph DFS走訪, 建無向相鄰矩陣.cpp}

\section{Graph DFS剪枝回朔, 狀態空間搜尋}

\lstinputlisting{C:/Users/ae887/OneDrive/桌面/codebook/code/Graph DFS剪枝回朔, 狀態空間搜尋.cpp}

\section{Graph DFS回朔找尋拓樸排序, 建相鄰有向邊}

\lstinputlisting{C:/Users/ae887/OneDrive/桌面/codebook/code/Graph DFS拓樸排序, DFS建相鄰有向邊.cpp}

\section{Graph DFS計算圖連接性}

\lstinputlisting{C:/Users/ae887/OneDrive/桌面/codebook/code/Graph DFS計算圖連接性.cpp}

\section{Graph BFS計算圖連接性}

\lstinputlisting{C:/Users/ae887/OneDrive/桌面/codebook/code/Graph BFS計算圖連接性.cpp}

\section{Graph 有向邊並查集, 速通性檢查, 樹判斷}

\lstinputlisting{C:/Users/ae887/OneDrive/桌面/codebook/code/Graph 有向邊並查集, 速通性檢查, 樹判斷.cpp}

\section{MST Kuskal計算最小樹新增無向邊權和}

\lstinputlisting{C:/Users/ae887/OneDrive/桌面/codebook/code/MST Kuskal計算最小樹新增無向邊權和.cpp}

\section{MST prim計算權和, 線性掃描最小邊, 密稠圖}

\lstinputlisting{C:/Users/ae887/OneDrive/桌面/codebook/code/MST prim計算權和, 線性掃描最小邊.cpp}

\section{SP Warshell閉包遞移, 二分法計算最長邊最小路徑}

\lstinputlisting{C:/Users/ae887/OneDrive/桌面/codebook/code/SP Warshell閉包遞移, 二分法計算最長邊最小路徑.cpp}

\section{SP Dijkstra重邊判斷, 找最短路徑}

\lstinputlisting{C:/Users/ae887/OneDrive/桌面/codebook/code/SP Dijkstra重邊判斷, 找最短路徑.cpp}

\section{SP Dijkstra二分搜尋最佳初始值}

\lstinputlisting{C:/Users/ae887/OneDrive/桌面/codebook/code/SP Dijkstra二分搜尋最佳初始值.cpp}

\section{SP SPFA求負權最短路徑}

\lstinputlisting{C:/Users/ae887/OneDrive/桌面/codebook/code/SP SPFA求負權最短路徑.cpp}

\section{BG HA(匈牙利算法)二分圖最大匹配}

\lstinputlisting{C:/Users/ae887/OneDrive/桌面/codebook/code/BG HA(匈牙利算法)二分圖最大匹配.cpp}

\section{BG 最大匹配數求邊覆蓋}

\lstinputlisting{C:/Users/ae887/OneDrive/桌面/codebook/code/BG 最大匹配數求邊覆蓋.cpp}

\section{BG 二分圖匹配最大化最小值}

\lstinputlisting{C:/Users/ae887/OneDrive/桌面/codebook/code/SP SPFA求負權最短路徑.cpp}

\section{BG KM求二分圖最小權和(負邊)}

\lstinputlisting{C:/Users/ae887/OneDrive/桌面/codebook/code/BG KM求二分圖最小權和(負邊).cpp}

\section{Flow EK求最大流}

\lstinputlisting{C:/Users/ae887/OneDrive/桌面/codebook/code/Flow EK求最大流.cpp}

\section{Flow SPFA求最小費用流, 帶權二分圖轉網路圖}

\lstinputlisting{C:/Users/ae887/OneDrive/桌面/codebook/code/Flow SPFA求最小費用流, 帶權二分圖轉網路圖.cpp}

\end{multicols}

\end{document}
